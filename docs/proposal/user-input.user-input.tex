The user will be able to interact with our program in the following ways.

{
  \newcommand{\mc}[1]{\multicolumn{2}{c}{#1}}
  \begin{longtabu} to 0.9\linewidth {c@{~~or}cX<{\strut}}
    \toprule
    \mc{\textbf{Keystroke}} & \textbf{Response} \\
    \midrule
    \endhead
    \bottomrule
    \mc{} & \vspace*{-5pt}\hfill\textit{Continued on next page\ldots} \\
    \endfoot
    \bottomrule
    \endlastfoot
    \textbf{W} & $\boldsymbol{\uparrow}$     & Move the camera ``to the north''. The notion of upwards may depend on how the user has rotated his or her view. \\
    \textbf{A} & $\boldsymbol{\leftarrow}$   & Move the camera ``to the west''. \\
    \textbf{S} & $\boldsymbol{\downarrow}$   & Move the camera ``to the south''. \\
    \textbf{D} & $\boldsymbol{\rightarrow}$  & Move the camera ``to the east''. \\
    \mc{scrollwheel}                    & Zoom in or out (basically, move the camera upwards or away from the floor to zoom out and downwards or to the floor to zoom in). \\
    \mc{\textbf{C}}                     & Start building conveyor belts (only in \textit{building mode}, see below). \\
    \mc{\textbf{R}}                     & Rotate the selected blocks $90$ degrees in clockwise direction, in place (only in \textit{building mode}). See below for a more detailed explanation. \\
    \mc{\textbf{Del}}                   & Delete the selected blocks (only in \textit{building mode}). \\
    \mc{\textbf{Alt} + \textbf{Enter}}  & Toggle full screen mode. \\
    \mc{\textbf{Esc}}                   & Exit the program. \\
  \end{longtabu}
}

\noindent Instead of using the keys mentioned above to move the camera around, the user may also move the mouse pointer to one of the four edges of the screen to move the camera in the corresponding position. We think that this is intuitive for most users.