During the implementation phase, we made some design decisions that we document in this section. This also includes interesting implementation details, for example solutions to some non-trivial implementation issues.

\subsection{Simulating luggage movement}
We initially implemented a simulation module ourselves, that was able to move luggage over the conveyor belts, but it did not support rotations. This means that luggage did not for example rotate when climbing an ascending conveyor belt or when it dropped of a belt. Furthermore the system was not really realistic, since we only used a very simple physics model. This meant that there was no collision detection between luggage items at all, and one piece of luggage could be placed inside another one.

After searching information about how to implement a more realistic physics system, we found out that it is extremely difficult to do so, and we decided to use an external library for the physics, named \emph{Bullet} (more accurately, a Java port of Bullet called \emph{JBullet}). Now we had to integrate JBullet in our application, which was not too difficult to do.

Initially, we had a little problem with getting the conveyor belts to actually move in JBullet. The belts were animated, but luggage would not move when it would fall on a conveyor belt. We found ways of simulating conveyor belts in Bullet, but unfortunately we were not able to use this since JBullet, the Java port we were using, was a couple of versions behind on Bullet. Finally, we found a way that worked in JBullet as well. Bullet has a notion of static objects. These are objects without mass that cannot (and will not) move. It is however possible to give these objects a linear and angular velocity, which results in dynamic objects hitting that static object taking over those velocities. This is precisely what we could use to simulate conveyor belts. We even managed to simulate curved conveyor belts realistically using a combination of a diagonal linear velocity and a non-zero angular velocity.