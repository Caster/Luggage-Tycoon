\documentclass[a4paper,10pt]{article}
\usepackage[utf8]{inputenc}
\usepackage[T1]{fontenc}
\usepackage{amsmath}

\usepackage{booktabs}

\usepackage{longtable}
\usepackage{tabu}

\usepackage{xspace}

\usepackage[sc]{mathpazo}
\linespread{1.05}

\newcommand*{\accgtitle}{Building a conveyor belt system editor \& simulator: Change report\xspace}
\newcommand*{\accgversion}{Draft, v2 $\leftrightarrow$ Final\xspace}

\usepackage[hidelinks]{hyperref}
\hypersetup{pdftitle={2IV05: \accgtitle}, pdfauthor={Thom Castermans \& Willem Sonke}}
\usepackage[format=hang]{caption}

\usepackage{graphicx}
\graphicspath{ {img/} }

\usepackage{geometry}

\usepackage[usenames,dvipsnames,svgnames,table]{xcolor}

\usepackage{changepage}

\usepackage[nottoc,notlof,notlot]{tocbibind} 
\tocotherhead{subsection}

\renewcommand{\labelitemi}{$\bullet$}
\newcommand{\eo}[2]{#1\,/\,#2}

\begin{document}

\definecolor{accgblue}{RGB}{97,147,207}
\begin{adjustwidth}{0.05\textwidth}{0.05\textwidth}  
  \textcolor{accgblue}{\rule{0.9\textwidth}{0.8pt}}
  
  \begin{center}
    \large 2IV05 -- Additional component computer graphics\\
    \Large\textbf{\accgtitle}\\[10pt]
    \normalsize Thom Castermans \qquad Willem Sonke
  \end{center}
  
  \vspace{10pt}\noindent\footnotesize~~Eindhoven University of Technology\hfill\accgversion\ -- \today~~
  
  \vspace{-6pt}\noindent\textcolor{accgblue}{\rule{0.9\textwidth}{0.8pt}}
\end{adjustwidth}

\begin{abstract}
 This document details the changes between the second draft version of the report (version ``Draft, v2'' of 14 March 2014) and the final version (version ``Final'' of 22 April 2014).
\end{abstract}

\section*{Change report}
The changes are given in the table below.

\begin{longtabu}{r>{\raggedright}X<{\strut}}
  \toprule
  \textbf{Section} & \textbf{Improvement} \\
  \midrule
  \endhead
  \bottomrule
  & \vspace*{-5pt}\hfill\textit{Continued on next page\ldots} \\
  \endfoot
  \bottomrule
  \endlastfoot
  1.1 & Determined $\rightarrow$ determine \\
  1.1.1 & Improved wording in the last paragraph about the program being fault-preventing \\
  2.1 & Added a new section (2.1.1) to explain the problem of texture mapping \\
  2.1 & Added more explanation about the solution of the texture mapping problem \\
  2.3 & Added a figure to explain Bresenham's algorithm \\
  3.1 & Rewritten to better explain how JBullet is used \\
  3.2 & Added this section to explain more on moving conveyor belts in JBullet \\
  5 & Removed the section (previously 5.1) on things that were still to do \\
  5 & Expanded the results section to the newly implemented features \\
  6 & Added the conclusion \\
  7 & Added the future work \\
\end{longtabu}


\end{document}
