In this section, we divide the problem into subproblems, and analyse them separately.

We can divide the requirements into the two following subproblems.
\begin{itemize}
 \item \emph{The editor.} We need to show the conveyor belt system, and enable the user to edit it by adding new conveyor belts and removing them. This includes a GUI for the user to for example choose blocks.
 \item \emph{The simulator.} We need to simulate and show movement of luggage over the conveyor belts.
\end{itemize}
We can subdivide the editor again in several parts.
\begin{itemize}
 \item \emph{Viewing conveyor belts.} We need to show the conveyor belt system to the user in 3D.
 \item \emph{The interface.} We need to have an interface with which the user can choose different kinds of blocks, rotate them, and so on.
 \item \emph{Placing blocks.} Finally, the user needs to have a way to indicate where blocks should be placed or removed; therefore, we need to be able to convert 2D mouse clicks to the corresponding 3D location.
\end{itemize}

\subsection{Viewing}
\textbf{TO DO schetsen van de verschillende soorten banden en hoe ze eruit zien -- of dit in de vorige sectie misschien?}

\textbf{TO DO het aan elkaar tekenen van banden.}

\subsection{Interface}
\textbf{TO DO}

\subsection{Placing}
\textbf{@Thom: hier iets over dat lijn-in-cel-algoritme?}

\subsection{Simulation}