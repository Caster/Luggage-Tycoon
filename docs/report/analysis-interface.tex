We discussed in Section~\ref{subsubsec:design-goals-editor} what we expect from the interface of the program. Since we plan to implement our tool in Java, using a thin wrapper\footnote{LWJGL to be precise, see \url{http://lwjgl.org/}.} around OpenGL, we will have to use some kind of GUI toolkit, or write our own. We did a quick search and did not find any useful toolkits, so we will write our own (basic) GUI toolkit. This toolkit should be able to render components in OpenGL, using the 2D OpenGL capabilities to render a GUI on top of the 3D scene. It should furthermore be easily extendable, flexible in the sense that placements of menu bars should be configurable for example and finally it must be capable of handling input events quickly and without complex API. Since we want everything to look good, the toolkit should support buttons (in menu bars) with icons and a configurable font. Preferably, the toolkit also supports animations. For example, when a button is hovered, its background colour may change. This change might be animated, simply because it looks good.

If we add tooltips to our menu items and buttons, the toolkit must be of course able to render those. Preferably, tooltips are shown and hidden with a fading animation, but this is not a hard requirement. In case the toolkit does support animations at all, it would however be good if it does so in a as-generic-as-possible way, so that the functionality can be used for different purposes (hover animations, tooltip animations, \ldots).