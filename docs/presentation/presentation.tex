\documentclass[12pt, aspectratio=169]{beamer} % aspectratio = 43 of 169

% kleuren (theme=...): red (standaard), blue, cyan, orange, green
% official=false: een mooie, maar niet-officiele stijl
% official=true: de stijl die ze ook in PowerPoint hebben, met dat blauwe blokje onderaan
% MERK OP: bij official=true is de nummering van de slides verkeerd!
\usetheme[department=winuk,official=false,theme=cyan,innovation=false,titlebgimage=imgs/screenshot-titleframe.png]{tue2008}

\mode<presentation>

\setbeamercolor{alerted text}{fg=tueblue}

\graphicspath{{./imgs}}

\title{Building a conveyor belt editor and simulator}
\author{Thom Castermans and Willem Sonke}

\begin{document}

\begin{titleframe}
\end{titleframe}

\begin{frame}
  \transduration<1>{3}
  \frametitle{Project idea}
  \begin{itemize}
    \item Block editor\uncover<2->{\ldots\ with a twist!
    \item Building a conveyor belt system
    \item Simulating luggage movement}
  \end{itemize}
\end{frame}

\begin{frame}
  \frametitle{Work so far}
  \begin{itemize}
    \item Using LWJGL to address OpenGL from Java
    \item Building a GUI in OpenGL
    \item Intuitive camera control
    \item Animating conveyor belts
    \item Initial simulation
  \end{itemize}
\end{frame}

\begin{frame}
  \frametitle{Planned work}
  \begin{itemize}
    \item Improving simulation
    \item Actual editing\ldots
    \item Adding scanners
    \item Game-like experience, with levels
    \item ``Sandbox mode''
  \end{itemize}
\end{frame}


\begin{frame}
 \frametitle{Demo}
 \alert{Demo time!}
\end{frame}

\end{document}
